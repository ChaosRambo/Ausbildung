\documentclass[10pt,a4paper]{article}
\usepackage[utf8]{inputenc}
\usepackage[german]{babel}
\usepackage[T1]{fontenc}
\usepackage{amsmath}
\usepackage{amsfonts}
\usepackage{amssymb}
\usepackage{graphicx}

\usepackage{tikz}
\usetikzlibrary{mindmap,trees}

\author{Daniel Thümen}
\title{Ethernetstandard des IEEE}
\begin{document}
\section*{Aufgaben zum Aufgabenblatt}
\begin{enumerate}
\item Stellen Sie alle physikalischen Übertragungsmedien, die im Text erwähnt werden, strukturiert mit einer Mind-Map dar.
\end{enumerate}
\begin{tikzpicture}
\path[mindmap, grow cyclic, every node/.style=concept, concept color=orange!40, 
    level 1/.append style={level distance=5.7cm,sibling angle=90},
    level 2/.append style={level distance=3cm,sibling angle=90},]
	node[concept] {Physikalische Übertragungsmedien}
	[clockwise from=0]
	child[concept color=green!50!black] {
      node[concept] {10Base5}
      [clockwise from=90]
      child { node[concept] {MRAM} }
    }
    child[concept color=red] {
      node[concept] {1Base5 StarLAN}
      [clockwise from=-120]
      child { node[concept] {NVSRAM} }
    }
    child[concept color=yellow] {
      node[concept] {10Base-T}
      [clockwise from=-120]
      child { node[concept] {NVSRAM} }
    }
    child[concept color=orange] {
      node[concept] {10Base-F}
      [clockwise from=-120]
      child { node[concept] {NVSRAM} }
    };
\end{tikzpicture}
\end{document}